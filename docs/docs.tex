%! suppress = TooLargeSection
%! suppress = SentenceEndWithCapital
%! suppress = TooLargeSection
% Preamble
\documentclass[11pt]{PyRollDocs}
\usepackage{textcomp}
\usepackage{csquotes}
\usepackage{wasysym}

\addbibresource{refs.bib}
% Document
\begin{document}

    \title{Tension spreading interaction PyRolL Plugin}
    \author{Christoph Renzing}
    \date{\today}

    \maketitle

    The PyRolL plugin pyroll-tension-spreading-interaction calculates the tension influenced spread of the rolled profile.
    The model was published by \textcite{Dobler1998, MaukDobler1999} using data from \textcite{Nikkilä1977} and \textcite{Treis1968}.


    \section{Model approach}\label{sec:model-approach}
    To incorporate the effect of longitudinal tensions into the process of groove rolling, \textcite{Dobler1998} developed an empirical model derived from different measurements by \textcite{Nikkilä1977} and \textcite{Treis1968}.
    The model is applied by defining a \enquote{tension free logarithmic elongation} as well as a \enquote{tension influenced logarithmic elongation}.
    Both values are combined using superposition according to \autoref{eq:tension-superposition}

    \begin{equation}
        \varphi_l = \varphi_{l, \sigma=0} + \Delta\varphi_{l, \sigma}
        \label{eq:tension-superposition}
    \end{equation}

    The tension influenced logarithmic elongation $\Delta\varphi_{l, \sigma}$ is treated as a function of the mean flow stress ($k_{f,m}$), the acting tension stresses ($\sigma_{l,0}$,$\sigma_{l,1}$) as well as cross-section change.
    $\sigma_{l,0}$ is the acting mean back tension and $\sigma_{l,1}$ the mean front tension.
    Another influencing parameter which is treated indirectly is the strain rate $\dot{\varphi}$.
    This value is incorporated by the flow stress.
    Therefore,to achieve better calculative results a flow stress model incorporating the strain rate should be chosen.
    The influence of friction is incorporated by the drought $\epsilon_h$ as well as the roll gap ratio $\frac{b_0}{h_0}$ as well as the ratio of contact area and mean profile cross-section $\frac{A_d}{A_m}$.

    \begin{gather}
        k_{f,m} = \frac{k_{f,0} + 2 k_{f,1} }{3} \\
        x_0 = \frac{\sigma_{l, 0}}{k_{f,m}}\\
        x_1 = \frac{\sigma_{l, 1}}{k_{f,m}}\\
        A_m = \frac{A_0 + 2 A_1 }{3} \\
        \Delta\varphi_{l, \sigma} = f \left( x_0, x_1, \epsilon_h, \frac{b_0}{h_0}m \frac{A_d}{A_m} \right)
    \end{gather}

    From measurements by \textcite{Nikkilä1977} and \textcite{Treis1968}, Dobler~\cite{Dobler1998} came to the conclusion that the backward tension $x_0$ has a quadratic influence and the influence of the front tension $x_1$ could be model by a linear approach.
    He therefore published the following equation for calculation of the tension influenced elongation:

    \begin{equation}
        \Delta \varphi_{l, \sigma} = k_1 x_0^2 + k_2 x_0 + k_3 x_1
        \label{eq:dobler-model}
    \end{equation}

    The coefficients included in this formula are functions of the geometric parameters and are calculated by a linear relationship.

    \begin{equation}
        k_i = m_{i,1} \epsilon_h + m_{i,2} \frac{b_0}{h_0} + m_3 \frac{A_d}{A_m}
        \label{eq:dobler-coefficients}
    \end{equation}

    The values for the coefficients $m_{i,j}$ where calculated by \textcite{Dobler1998} using an equilibrium calculation for overdetermined linear systems of equations.

    \begin{table}
        \centering
        \caption{Coefficients for calculation of the tension influenced elongation $\varphi_{l, \sigma}$.}
        \label{tab:dobler-coefficients}
        \begin{tabular}{llll}
            \toprule
            i & $m_{i,1}$ & $m_{i,2}$   & $m_{i,3}$ \\
            \midrule
            1 & 1.05502   & 0.100816    & -0.591029 \\
            2 & -0.886507 & -0.00258613 & 0.159971  \\
            3 & -0.347681 & -0.0457338  & 0.0525161 \\
            \bottomrule
        \end{tabular}
    \end{table}

    The statistical evaluation of calculated and measured forming influenced elongations shows a determination of 92.91\%.


    \section{Usage instructions}\label{sec:usage-instructions}
    The plugin can be loaded under the name \texttt{pyroll\_tension\_spreading\_interaction}.

    An implementation of the \lstinline{log_elongation} hook on \lstinline{RollPass} is provided.
    Furthermore, an implementation of the \lstinline{width} hook on \lstinline{RollPass.OutProfile} is provided.

    Additionally, hooks on \lstinline{RollPass} are defined, which are used in the calculation, as listed in \autoref{tab:hookspecs}.

    \begin{table}
        \centering
        \caption{Hooks specified by this plugin.}
        \label{tab:hookspecs}
        \begin{tabular}{ll}
            \toprule
            Hook name                               & Meaning                                                 \\
            \midrule
            \texttt{tension\_model}                 & Tension model of Dobler and Mauk                        \\
            \texttt{log\_elongation\_with\_tension} & Mean back tension of the roll pass $\varphi_{l,\sigma}$ \\
            \bottomrule
        \end{tabular}
    \end{table}

    \printbibliography


\end{document}