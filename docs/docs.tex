%! suppress = TooLargeSection
%! suppress = SentenceEndWithCapital
%! suppress = TooLargeSection
% Preamble
\documentclass[11pt]{PyRollDocs}
\usepackage{textcomp}
\usepackage{csquotes}
\usepackage{wasysym}

\addbibresource{refs.bib}
% Document
\begin{document}

    \title{Karman Power and Labour PyRolL Plugin}
    \author{Christoph Renzing}
    \date{\today}

    \maketitle

    The PyRolL plugin pyroll-tension-spreading-interaction calculates the tension influenced spread of the rolled profile.
    The model was published by \textcite{Dobler1998, MaukDobler1999} using data from \textcite{Nikkilä1977} and \textcite{Treis1968}.


    \section{Model approach}\label{sec:model-approach}
    To incorporate the effect of longitudinal tensions into the process of groove rolling, \textcite{Dobler1998} developed an empirical model derived from different measurements by \textcite{Nikkilä1977} and \textcite{Treis1968}.
    The model is applied by defining a \enquote{tension free logarithmic elongation} as well as a \enquote{tension influenced logarithmic elongation}.
    Both values are combined using superposition according to \autoref{eq:tension-superposition}

    \begin{equation}
        \varphi_l = \varphi_{l, \sigma=0} + \Delta\varphi_{l, \sigma}
        \label{eq:tension-superposition}
    \end{equation}

    The tension influenced logarithmic elongation $\Delta\varphi_{l, \sigma}$ is treated as a function (\autoref{eq:tension-function}) of the mean flow stress ($k_{f,m}$), the acting tension stresses ($\sigma_{l,0}$,$\sigma_{l,1}$) as well as cross-section change.
    $\sigma_{l,0}$ is the acting mean back tension and $\sigma_{l,1}$ the mean front tension.
    Another influencing parameter which is treated indirectly is the strain rate $\dot{\varphi}$.
    This value is incorporated by the flow stress.
    Therefore,to achieve better calculative results a flow stress model incorporating the strain rate should be chosen.

    \begin{gather}
        x_0 = \frac{\sigma_{l, 0}}{k_{f,m}}\\
        x_1 = \frac{\sigma_{l, 1}}{k_{f,m}}\\
        \Delta\varphi_{l, \sigma} = f \left( x_0, x_1, \epsilon_h, \frac{b_0}{h_0}m \frac{A_d}{A_m} \right)
    \end{gather}


    \section{Usage instructions}\label{sec:usage-instructions}
    The plugin can be loaded under the name \texttt{pyroll\_karman\_power\_and\_labour}.

    An implementation of the \lstinline{roll_force} and \lstinline{mean_neutral_plane_position} hook on \lstinline{RollPass} is provided.
    Furthermore, an implementation of the \lstinline{roll_torque} hook on \lstinline{RollPass.Roll} is provided.

    Additionally, hooks on \lstinline{RollPass} are defined, which are used in the calculation, as listed in \autoref{tab:hookspecs}.
    The hooks \lstinline{mean_front_tension}, \lstinline{mean_back_tension} and \lstinline{coulomb_friction_coefficient} have to set and adjusted individually.

    \begin{table}
        \centering
        \caption{Hooks specified by this plugin.}
        \label{tab:hookspecs}
        \begin{tabular}{ll}
            \toprule
            Hook name                               & Meaning                                                \\
            \midrule
            \texttt{coulomb\_friction\_coefficient} & Coulomb's friction coefficient $\mu$                   \\
            \texttt{mean\_back\_tension}            & Mean back tension of the roll pass $\sigma_{x,0}$      \\
            \texttt{mean\_front\_tension}           & Mean front tension of the roll pass $\sigma_{x,1}$     \\
            \texttt{mean\_neutral\_plane\_position} & Mean neutral plane postion $x_N$                       \\
            \texttt{karman\_solution}               & KarmanSolver object with solution values as attributes \\
            \bottomrule
        \end{tabular}
    \end{table}

    \printbibliography


\end{document}